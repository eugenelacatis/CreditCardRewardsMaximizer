% Options for packages loaded elsewhere
\PassOptionsToPackage{unicode}{hyperref}
\PassOptionsToPackage{hyphens}{url}
\documentclass[
  twocolumn]{article}
\usepackage{xcolor}
\usepackage[margin=0.75in]{geometry}
\usepackage{amsmath,amssymb}
\setcounter{secnumdepth}{-\maxdimen} % remove section numbering
\usepackage{iftex}
\ifPDFTeX
  \usepackage[T1]{fontenc}
  \usepackage[utf8]{inputenc}
  \usepackage{textcomp} % provide euro and other symbols
\else % if luatex or xetex
  \usepackage{unicode-math} % this also loads fontspec
  \defaultfontfeatures{Scale=MatchLowercase}
  \defaultfontfeatures[\rmfamily]{Ligatures=TeX,Scale=1}
\fi
\usepackage{lmodern}
\ifPDFTeX\else
  % xetex/luatex font selection
\fi
% Use upquote if available, for straight quotes in verbatim environments
\IfFileExists{upquote.sty}{\usepackage{upquote}}{}
\IfFileExists{microtype.sty}{% use microtype if available
  \usepackage[]{microtype}
  \UseMicrotypeSet[protrusion]{basicmath} % disable protrusion for tt fonts
}{}
\makeatletter
\@ifundefined{KOMAClassName}{% if non-KOMA class
  \IfFileExists{parskip.sty}{%
    \usepackage{parskip}
  }{% else
    \setlength{\parindent}{0pt}
    \setlength{\parskip}{6pt plus 2pt minus 1pt}}
}{% if KOMA class
  \KOMAoptions{parskip=half}}
\makeatother
\setlength{\emergencystretch}{3em} % prevent overfull lines
\providecommand{\tightlist}{%
  \setlength{\itemsep}{0pt}\setlength{\parskip}{0pt}}
\usepackage{bookmark}
\IfFileExists{xurl.sty}{\usepackage{xurl}}{} % add URL line breaks if available
\urlstyle{same}
\hypersetup{
  pdftitle={Credit Card Rewards Maximizer: AI-Powered Optimization System},
  hidelinks,
  pdfcreator={LaTeX via pandoc}}

\title{Credit Card Rewards Maximizer: AI-Powered Optimization System}
\author{}
\date{}

\begin{document}
\maketitle

\subsection{Authors}\label{authors}

\textbf{Eugene Lacatis}\\
Master's in Software Engineering\\
San Jose State University\\
eugene.lacatis@sjsu.edu

\textbf{Irwin Salamanca}\\
Master's in Software Engineering\\
San Jose State University\\
irwin.salamanca@sjsu.edu

\textbf{Matt Tang}\\
Master's in Software Engineering\\
San Jose State University\\
matthew.tang@sjsu.edu

\textbf{Atharva Prasanna Mokashi}\\
Master's in Software Engineering\\
San Jose State University\\
atharvaprasanna.mokashi@sjsu.edu

\begin{center}\rule{0.5\linewidth}{0.5pt}\end{center}

\subsection{Abstract}\label{abstract}

The Credit Card Rewards Maximizer is an AI-powered mobile application
designed to help users maximize their credit card rewards by
recommending the optimal card for every purchase. Leveraging Llama 3 AI
through Groq's API and built with FastAPI and React Native, the system
analyzes user credit card portfolios in real-time to provide
personalized recommendations based on merchant, amount, category, and
optimization goals. The application features comprehensive analytics,
transaction tracking, and agentic AI capabilities that learn from user
behavior to improve recommendations over time. This paper details the
system architecture, implementation, testing strategies, and
technologies employed in building this intelligent financial
optimization tool.

\textbf{Index Terms:} Credit Card Optimization, Artificial Intelligence,
Llama 3, Rewards Maximization, Financial Technology, Mobile Application,
FastAPI, React Native, PostgreSQL, Groq API, Machine Learning,
Personalized Recommendations, Transaction Analytics, Agentic AI,
LangChain, Docker, Cloud Computing.

\begin{center}\rule{0.5\linewidth}{0.5pt}\end{center}

\subsection{I. Introduction}\label{i.-introduction}

In today's consumer landscape, credit card rewards programs have become
increasingly complex, with multiple cards offering varying rewards rates
across different spending categories. The average American owns 3-4
credit cards, yet most consistently use only one or two, leaving
significant rewards value unclaimed. Manually tracking which card offers
the best rewards for each purchase is mentally exhausting and
impractical for daily transactions.

The Credit Card Rewards Maximizer addresses this challenge by providing
an intelligent, AI-powered solution that instantly recommends the
optimal credit card for every purchase. By analyzing the user's credit
card portfolio, transaction details, and personal optimization goals
(cash back, travel points, or balanced rewards), the system ensures
users never miss an opportunity to maximize their rewards.

The application combines modern mobile technology with advanced AI to
deliver recommendations in under 2 seconds, making optimal card
selection effortless. Beyond simple recommendations, the system provides
comprehensive analytics showing total savings, best-performing cards,
spending patterns, and missed opportunities, empowering users to make
data-driven financial decisions.

\subsection{II. System Architecture}\label{ii.-system-architecture}

The Credit Card Rewards Maximizer employs a modern, scalable
microservices architecture designed for high performance and
reliability. The system is containerized using Docker and can be
deployed across multiple environments.

\subsubsection{A. Frontend (React Native with
Expo)}\label{a.-frontend-react-native-with-expo}

The frontend is built using React Native with Expo, enabling true
cross-platform development for iOS, Android, and web from a single
codebase. The application features:

\begin{itemize}
\tightlist
\item
  \textbf{Transaction Input Screen}: Allows users to enter merchant
  name, purchase amount, spending category, and optimization goal
\item
  \textbf{Card Management Screen}: Enables users to add, edit, and
  manage their credit card portfolio
\item
  \textbf{Recommendations Display}: Shows the optimal card with detailed
  explanations and expected rewards
\item
  \textbf{Analytics Dashboard}: Visualizes savings trends, category
  breakdowns, and performance metrics
\end{itemize}

The frontend communicates with the backend via RESTful APIs, with all
network requests handled through a centralized API service layer that
manages authentication, error handling, and response parsing.

\subsubsection{B. Backend (FastAPI)}\label{b.-backend-fastapi}

The backend is powered by FastAPI, a modern Python web framework known
for its high performance and automatic API documentation. Key features
include:

\begin{itemize}
\tightlist
\item
  \textbf{RESTful API Endpoints}: Comprehensive API for card management,
  recommendations, transactions, and analytics
\item
  \textbf{Automatic Documentation}: Interactive API documentation via
  Swagger UI at \texttt{/docs}
\item
  \textbf{Async Support}: Asynchronous request handling for improved
  performance
\item
  \textbf{CORS Middleware}: Configured for cross-origin requests from
  mobile clients
\item
  \textbf{Health Monitoring}: Health check endpoints for deployment
  monitoring
\item
  \textbf{Observability}: Integrated Prometheus metrics for tracking
  system performance and user engagement
\end{itemize}

The backend architecture follows clean separation of concerns with
dedicated modules for database operations (CRUD), AI agents, data
models, and API routes.

\subsubsection{C. Database (PostgreSQL)}\label{c.-database-postgresql}

PostgreSQL serves as the primary relational database, providing:

\begin{itemize}
\tightlist
\item
  \textbf{User Management}: Stores user profiles and authentication data
\item
  \textbf{Credit Card Data}: Maintains card details including rewards
  rates, benefits, and issuer information
\item
  \textbf{Transaction History}: Records all transactions with
  recommendations and actual card usage
\item
  \textbf{Analytics Data}: Aggregates data for performance metrics and
  insights
\item
  \textbf{Behavioral Patterns}: Tracks user preferences and spending
  patterns for personalized recommendations
\end{itemize}

The database schema uses SQLAlchemy ORM for type-safe database
operations and includes proper indexing for frequently queried fields.

\subsubsection{D. AI Engine (Groq + Llama 3 +
LangChain)}\label{d.-ai-engine-groq-llama-3-langchain}

The AI recommendation engine combines multiple technologies:

\begin{itemize}
\tightlist
\item
  \textbf{Groq API}: Provides ultra-fast inference for Llama 3 models
\item
  \textbf{Llama 3}: Open-source large language model for natural
  language understanding and reasoning
\item
  \textbf{LangChain}: Framework for building AI agent systems with
  memory and tool use
\item
  \textbf{Agentic System}: Multi-agent architecture including behavior
  analysis, proactive suggestions, context awareness, planning,
  learning, and automation agents
\end{itemize}

The AI system analyzes transaction context, user card portfolio, and
historical behavior to generate personalized recommendations with
confidence scores and detailed explanations.

\subsection{III. Functionalities}\label{iii.-functionalities}

The Credit Card Rewards Maximizer provides comprehensive features for
rewards optimization:

\subsubsection{A. AI-Powered Card
Recommendations}\label{a.-ai-powered-card-recommendations}

The core functionality analyzes each transaction to recommend the
optimal credit card:

\begin{itemize}
\tightlist
\item
  \textbf{Real-time Analysis}: Processes merchant, amount, category, and
  optimization goal
\item
  \textbf{Multi-factor Evaluation}: Considers cash back rates, points
  multipliers, annual fees, and card benefits
\item
  \textbf{Confidence Scoring}: Provides confidence levels for each
  recommendation
\item
  \textbf{Detailed Explanations}: AI-generated explanations of why a
  specific card was recommended
\item
  \textbf{Alternative Options}: Shows second and third-best card options
  with expected values
\item
  \textbf{Credit Score Integration}: Users provide their FICO score during registration, enabling the system to filter and display only credit cards they are likely to qualify for based on typical approval requirements
\end{itemize}

\subsubsection{B. Multi-Card Portfolio
Management}\label{b.-multi-card-portfolio-management}

Users can manage their entire credit card portfolio:

\begin{itemize}
\tightlist
\item
  \textbf{Card Addition}: Add cards with issuer, rewards structure,
  annual fees, and benefits
\item
  \textbf{Rewards Configuration}: Define category-specific cash back
  rates and points multipliers
\item
  \textbf{Card Activation/Deactivation}: Control which cards are
  considered for recommendations
\item
  \textbf{Credit Limit Tracking}: Monitor credit limits and utilization
\item
  \textbf{Card Details}: Store last four digits and other identifying
  information
\end{itemize}

\subsubsection{C. Transaction Tracking and
History}\label{c.-transaction-tracking-and-history}

Comprehensive transaction management:

\begin{itemize}
\tightlist
\item
  \textbf{Automatic Logging}: All recommendations are saved with
  transaction details
\item
  \textbf{Historical Analysis}: View past transactions with recommended
  vs.~actual card used
\item
  \textbf{Missed Opportunities}: Identify transactions where suboptimal
  cards were used
\item
  \textbf{Category Breakdown}: Analyze spending patterns across
  categories
\item
  \textbf{Merchant Tracking}: Track frequently visited merchants
\end{itemize}

\subsubsection{D. Analytics and
Insights}\label{d.-analytics-and-insights}

Data-driven insights for financial optimization:

\begin{itemize}
\tightlist
\item
  \textbf{Total Rewards Earned}: Track cumulative cash back and points
  earned
\item
  \textbf{Savings Visualization}: See potential vs.~actual savings over
  time
\item
  \textbf{Best Performing Cards}: Identify which cards generate the most
  value
\item
  \textbf{Category Analysis}: Understand spending distribution across
  categories
\item
  \textbf{Weekly Trends}: Monitor spending and rewards patterns over
  time
\item
  \textbf{Optimization Rate}: Measure how often optimal recommendations
  are followed
\end{itemize}

\subsubsection{E. Behavioral Learning and
Automation}\label{e.-behavioral-learning-and-automation}

Advanced agentic AI features:

\begin{itemize}
\tightlist
\item
  \textbf{Preference Learning}: System learns user preferences from
  feedback and behavior
\item
  \textbf{Proactive Suggestions}: Identifies missed optimization
  opportunities
\item
  \textbf{Automation Rules}: Create rules for automatic card selection
  based on conditions
\item
  \textbf{Context Awareness}: Considers location, time, and historical
  patterns
\item
  \textbf{Adaptive Recommendations}: Continuously improves based on user
  interactions
\end{itemize}

\subsection{IV. Persona}\label{iv.-persona}

The system is designed for multiple user types with different needs:

\subsubsection{A. Individual Consumers}\label{a.-individual-consumers}

Primary users who want to maximize their credit card rewards:

\begin{itemize}
\tightlist
\item
  \textbf{Use Case}: Daily purchase optimization across multiple credit
  cards
\item
  \textbf{Benefits}: Instant recommendations, automated tracking,
  comprehensive analytics
\item
  \textbf{Privacy}: All data remains private and secure
\item
  \textbf{Value}: Maximize rewards without mental overhead of tracking
  multiple card benefits
\end{itemize}

\subsubsection{B. Financial Enthusiasts}\label{b.-financial-enthusiasts}

Users who actively manage their finances and want detailed insights:

\begin{itemize}
\tightlist
\item
  \textbf{Use Case}: Deep analysis of spending patterns and optimization
  opportunities
\item
  \textbf{Benefits}: Advanced analytics, trend visualization, missed
  opportunity identification
\item
  \textbf{Features}: Historical analysis, category breakdowns,
  performance metrics
\item
  \textbf{Value}: Data-driven decision making for credit card portfolio
  optimization
\end{itemize}

\subsubsection{C. Credit Card Beginners}\label{c.-credit-card-beginners}

Users new to credit card rewards programs:

\begin{itemize}
\tightlist
\item
  \textbf{Use Case}: Learning which cards to use for different purchases
\item
  \textbf{Benefits}: Educational explanations, simple recommendations,
  guided setup
\item
  \textbf{Features}: Clear reasoning for recommendations, benefit
  explanations
\item
  \textbf{Value}: Simplified rewards optimization without requiring
  expert knowledge
\end{itemize}

\subsection{V. Technologies Used}\label{v.-technologies-used}

\subsubsection{A. React Native (Frontend
Development)}\label{a.-react-native-frontend-development}

React Native with Expo was chosen for frontend development due to:

\begin{itemize}
\tightlist
\item
  \textbf{Cross-Platform}: Single codebase for iOS, Android, and web
\item
  \textbf{Hot Reload}: Fast development iteration with instant updates
\item
  \textbf{Native Performance}: Near-native performance for mobile
  applications
\item
  \textbf{Rich Ecosystem}: Extensive library support and community
  resources
\item
  \textbf{Expo Tools}: Simplified build process and over-the-air updates
\end{itemize}

\subsubsection{B. FastAPI (Backend
Development)}\label{b.-fastapi-backend-development}

FastAPI provides a modern Python backend framework with:

\begin{itemize}
\tightlist
\item
  \textbf{High Performance}: Comparable to Node.js and Go in speed
\item
  \textbf{Automatic Documentation}: OpenAPI/Swagger documentation
  generation
\item
  \textbf{Type Safety}: Python type hints for better code quality
\item
  \textbf{Async Support}: Native asynchronous request handling
\item
  \textbf{Easy Testing}: Built-in testing utilities and fixtures
\end{itemize}

\subsubsection{C. PostgreSQL (Database
Management)}\label{c.-postgresql-database-management}

PostgreSQL was selected for its:

\begin{itemize}
\tightlist
\item
  \textbf{Reliability}: ACID compliance and data integrity
\item
  \textbf{JSON Support}: Native JSONB type for flexible schema elements
\item
  \textbf{Scalability}: Handles large datasets efficiently
\item
  \textbf{Advanced Features}: Full-text search, complex queries,
  indexing
\item
  \textbf{Open Source}: No licensing costs with enterprise-grade
  features
\end{itemize}

\subsubsection{D. Groq API with Llama 3}\label{d.-groq-api-with-llama-3}

The AI recommendation engine leverages:

\begin{itemize}
\tightlist
\item
  \textbf{Groq LPU}: Ultra-fast inference with Language Processing Units
\item
  \textbf{Llama 3}: State-of-the-art open-source language model
\item
  \textbf{Sub-2-Second Responses}: Real-time recommendations for user
  transactions
\item
  \textbf{Natural Language Understanding}: Contextual analysis of
  transactions
\item
  \textbf{Reasoning Capabilities}: Multi-factor decision making for
  optimal recommendations
\end{itemize}

\subsubsection{E. LangChain (AI Agent
Framework)}\label{e.-langchain-ai-agent-framework}

LangChain enables sophisticated AI agent systems:

\begin{itemize}
\tightlist
\item
  \textbf{Agent Architecture}: Multi-agent system for different
  recommendation aspects
\item
  \textbf{Memory Management}: Maintains context across interactions
\item
  \textbf{Tool Integration}: Connects AI to external data sources and
  APIs
\item
  \textbf{Prompt Engineering}: Structured prompts for consistent outputs
\item
  \textbf{Chain Composition}: Complex workflows combining multiple AI
  operations
\end{itemize}

\subsubsection{F. Docker
(Containerization)}\label{f.-docker-containerization}

Docker provides consistent deployment across environments:

\begin{itemize}
\tightlist
\item
  \textbf{Container Orchestration}: Docker Compose for multi-service
  setup
\item
  \textbf{Environment Isolation}: Separate containers for backend,
  frontend, and database
\item
  \textbf{Reproducible Builds}: Consistent environments across
  development and production
\item
  \textbf{Easy Deployment}: Single command to start entire application
  stack
\item
  \textbf{Resource Efficiency}: Lightweight containers compared to
  virtual machines
\end{itemize}

\subsubsection{G. SQLAlchemy (ORM)}\label{g.-sqlalchemy-orm}

SQLAlchemy provides database abstraction:

\begin{itemize}
\tightlist
\item
  \textbf{Type-Safe Queries}: Python objects mapped to database tables
\item
  \textbf{Migration Support}: Schema versioning and updates
\item
  \textbf{Relationship Management}: Automatic handling of foreign keys
  and joins
\item
  \textbf{Query Optimization}: Efficient query generation and execution
\item
  \textbf{Database Agnostic}: Easy switching between database backends
\end{itemize}

\subsection{VI. Testing}\label{vi.-testing}

The Credit Card Rewards Maximizer underwent comprehensive testing to
ensure reliability, performance, and correctness of both backend and
frontend components. The testing strategy employed pytest for backend
testing with a focus on unit tests, integration tests, and API endpoint
validation.

\subsubsection{A. Backend Testing
(pytest)}\label{a.-backend-testing-pytest}

pytest was used for comprehensive backend testing with the following
coverage:

\textbf{Unit Testing:} - Individual CRUD operations tested in isolation
- AI agent recommendation logic validation - Database model integrity
checks - Utility function correctness verification - Edge case handling
for invalid inputs

\textbf{Integration Testing:} - End-to-end API endpoint testing -
Database transaction integrity - AI service integration with Groq API -
Authentication and authorization flows - Error handling and recovery
mechanisms

\textbf{Test Coverage Areas:} - Card management endpoints (create, read,
update, delete) - Recommendation engine with various scenarios -
Analytics calculation accuracy - Transaction logging and retrieval -
User behavior tracking and learning - Automation rule execution

\textbf{Testing Approach:} - Fixtures for test data setup and teardown -
Mock external API calls (Groq) for consistent testing - Database
rollback after each test for isolation - Parametrized tests for multiple
input scenarios - Async test support for concurrent operations

\subsubsection{B. Frontend Testing}\label{b.-frontend-testing}

Frontend testing focused on manual verification and component
validation:

\textbf{Component Validation:} - Screen rendering verification across
iOS, Android, and Web - User interaction handling (taps, swipes, inputs)
- Navigation flow testing between tabs and stacks - Form input
validation and error state display - Responsive layout adjustments

\textbf{System Integration Testing:} - End-to-end testing using Expo Go
on physical devices - Network request handling and error recovery -
Loading state management during AI inference - Offline behavior and
reconnection handling - Real-time recommendation display verification

\subsection{VII. API Security}\label{vii.-api-security}

The Credit Card Rewards Maximizer implements multiple layers of security
to protect sensitive financial data:

\subsubsection{A. Environment Variables}\label{a.-environment-variables}

Sensitive configuration is managed through environment variables:

\begin{itemize}
\tightlist
\item
  \textbf{API Keys}: Groq API key stored in \texttt{.env} file
  (gitignored)
\item
  \textbf{Database Credentials}: PostgreSQL connection strings secured
\item
  \textbf{Secret Keys}: Application secrets for session management
\item
  \textbf{Configuration Isolation}: Separate configs for development and
  production
\end{itemize}

\subsubsection{B. CORS Configuration}\label{b.-cors-configuration}

Cross-Origin Resource Sharing is properly configured:

\begin{itemize}
\tightlist
\item
  \textbf{Allowed Origins}: Controlled list of permitted client origins
\item
  \textbf{Credential Support}: Secure cookie and authentication header
  handling
\item
  \textbf{Method Restrictions}: Limited to required HTTP methods
\item
  \textbf{Header Validation}: Only necessary headers allowed
\end{itemize}

\subsubsection{C. Input Validation}\label{c.-input-validation}

All user inputs are validated before processing:

\begin{itemize}
\tightlist
\item
  \textbf{Pydantic Models}: Type-safe request validation
\item
  \textbf{Field Constraints}: Min/max values, required fields, format
  validation
\item
  \textbf{SQL Injection Prevention}: Parameterized queries via
  SQLAlchemy ORM
\item
  \textbf{XSS Protection}: Input sanitization for stored data
\item
  \textbf{Amount Validation}: Positive values required for transactions
\end{itemize}

\subsubsection{D. Data Privacy}\label{d.-data-privacy}

User financial data is protected:

\begin{itemize}
\tightlist
\item
  \textbf{Database Encryption}: PostgreSQL supports encryption at rest
\item
  \textbf{HTTPS Communication}: All API calls encrypted in transit
  (Production)
\item
  \textbf{No Third-Party Sharing}: User data never shared with external
  services
\item
  \textbf{Minimal Data Collection}: Only necessary information stored
\item
  \textbf{User Data Ownership}: Users can export or delete their data
\end{itemize}

\subsubsection{E. Error Handling}\label{e.-error-handling}

Secure error responses:

\begin{itemize}
\tightlist
\item
  \textbf{Generic Error Messages}: No sensitive information in error
  responses
\item
  \textbf{Logging}: Detailed errors logged server-side only
\item
  \textbf{HTTP Status Codes}: Appropriate codes for different error
  types
\item
  \textbf{Graceful Degradation}: Fallback behavior when AI service
  unavailable
\end{itemize}

\subsection{VIII. Conclusion}\label{viii.-conclusion}

The Credit Card Rewards Maximizer successfully demonstrates the
application of modern AI technology to solve a real-world financial
optimization problem. By combining FastAPI for the backend, React Native
for cross-platform mobile development, PostgreSQL for reliable data
storage, and Llama 3 AI through Groq's API for intelligent
recommendations, the system delivers a comprehensive solution for
maximizing credit card rewards.

The platform offers a variety of functionalities including real-time
AI-powered card recommendations, comprehensive portfolio management,
transaction tracking, and detailed analytics. The agentic AI
architecture enables the system to learn from user behavior, provide
proactive suggestions, and continuously improve recommendation quality.
Users benefit from instant, data-driven decisions that eliminate the
mental overhead of tracking multiple card benefits while ensuring they
never miss an opportunity to maximize rewards.

The system underwent rigorous testing using pytest for backend
validation, achieving comprehensive coverage of API endpoints, database
operations, and AI integration. Security best practices were implemented
throughout, including environment variable management, input validation,
and secure data handling.

The technologies used in the project---React Native, FastAPI,
PostgreSQL, Groq API, Llama 3, LangChain, and Docker---provide a modern,
scalable, and maintainable foundation. The containerized architecture
ensures consistent deployment across environments, while the separation
of concerns enables independent scaling of frontend, backend, and
database components.

Future enhancements could include machine learning models for spending
prediction, integration with banking APIs for automatic transaction
import, social features for comparing optimization strategies, and
expanded support for additional card types and reward programs. The
Credit Card Rewards Maximizer delivers a practical, efficient, and
intelligent solution for consumers seeking to maximize their credit card
rewards through advanced technology.

\subsection{Acknowledgment}\label{acknowledgment}

We would like to express our sincere gratitude to all those who have
contributed to the successful completion of this project. Special thanks
to \textbf{Prof.~Rakesh Ranjan} for his guidance with the idea.

We also acknowledge the valuable suggestions from our peers, which have
significantly enhanced the quality of this work. Our appreciation
extends to the developers and engineers behind the technologies used in
this project, including React Native, FastAPI, PostgreSQL, Groq, Meta AI
(Llama 3), LangChain, and Docker, whose platforms and tools have been
fundamental in building this system.

Lastly, we thank the organizations and individuals who have supported
our work, providing the resources and environment to pursue this
research and development.

\end{document}
